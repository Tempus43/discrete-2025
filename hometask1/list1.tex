\documentclass{article}
\usepackage{amsmath, amsthm, amssymb, amsfonts}
\usepackage{thmtools}
\usepackage{graphicx}
\usepackage{indentfirst}
\usepackage{setspace}
\usepackage{geometry}
\usepackage{float}
\usepackage{hyperref}
\usepackage{cancel}
\usepackage[utf8]{inputenc}
\usepackage[russian]{babel}
\usepackage{framed}
\usepackage[dvipsnames]{xcolor}
\usepackage{tcolorbox}
\usepackage[textsize=small, textwidth=3.5cm]{todonotes}
\setlength{\marginparwidth}{1.5cm}

\colorlet{LightGray}{White!90!Periwinkle}
\colorlet{LightOrange}{Orange!15}
\colorlet{LightGreen}{Green!15}

\newcommand{\HRule}[1]{\rule{\linewidth}{#1}}

% Теоремные окружения и стили
\declaretheoremstyle[name=Theorem,]{thmsty}
\declaretheorem[style=thmsty,numberwithin=section]{theorem}
\tcolorboxenvironment{theorem}{colback=LightGray}

\declaretheoremstyle[name=Conjecture,]{thmsty}
\declaretheorem[style=thmsty,numberwithin=section]{conjecture}
\tcolorboxenvironment{conjecture}{colback=LightGray}

\declaretheoremstyle[name=Definition,]{thmsty}
\declaretheorem[style=thmsty,numberwithin=section]{definition}
\tcolorboxenvironment{definition}{colback=LightGray}

% Стиль задач: показываем пометку \NOTE в квадратных скобках и русское имя
\declaretheoremstyle[name=Задача,notebraces={[}{]}]{prosty}
\declaretheorem[style=prosty,numberlike=theorem]{task}

\declaretheoremstyle[name=Решение,notebraces={[}{]}]{prosty}
\declaretheorem[style=prosty, numberlike=definition]{solution}

\setstretch{1.2}
\geometry{
    textheight=9in,
    textwidth=5.5in,
    top=1in,
    headheight=12pt,
    headsep=25pt,
    footskip=30pt,
    right=4.5cm
}

\setuptodonotes{size=\small, color=white, bordercolor=white, textcolor=Bittersweet}

\title{Листочек №1 ``Теория множеств''}
\author{Стяжкин Артём Иванович, Матмех, группы 25.Б82-мм}
\date{Октябрь 2025}

\begin{document}
\maketitle

\section{Задачи - Решения}

% 1. Множества — эквивалентные формулировки
\begin{task}[1]
Доказать, что
\begin{enumerate}
    \item[а)] $A \subseteq B \cap C \Leftrightarrow A \subseteq B$ и $A \subseteq C$;
    \item[б)] $A \subseteq B \setminus C \Leftrightarrow A \subseteq B$ и $A \cap C = \varnothing$.
\end{enumerate}
\end{task}

\begin{solution}[1]
Решение, чтобы доказать это нужно подтвердить необходимость и достаточность каждого из утверждений:
\begin{enumerate}
	\item[a)] \begin{enumerate}
	\item[1.] $ \forall a \in A : a \in ( B \cap C ) \Rightarrow ( a \in B ), ( a \in C ) \Rightarrow A \subseteq B, A \subseteq C$;
	\item[2.] $ \forall a \in A : a \in B, a \in C \Rightarrow (a \in B), (a \in C) \Rightarrow A \subseteq (B \cap C)$.
	\end{enumerate}
	\item[b)] \begin{enumerate}
	\item[1.] $ \forall a \in A : a \in (B \setminus C) \Rightarrow (a \in B), (a \notin C) \Rightarrow A \subseteq B, A \cap C = \varnothing$;
	\item[2.] $ \forall a \in A : a \in B, a \notin C \Rightarrow (a \in B), (a \notin C) \Rightarrow A \subseteq (B \setminus C)$.
	\end{enumerate}
\end{enumerate}
\end{solution}

% 2. Мощности множеств — равенства и включения
\begin{task}[2]
Доказать следующие равенства и включения:
\begin{enumerate}
    \item[а)] $\mathcal{P}(A \cap B) = \mathcal{P}(A) \cap \mathcal{P}(B)$;
    \item[б)] $\mathcal{P}(A \cup B) \supseteq \mathcal{P}(A) \cup \mathcal{P}(B)$;
    \item[в)] $\mathcal{P}(A \setminus B) \subseteq (\mathcal{P}(A) \setminus \mathcal{P}(B)) \cup \{\varnothing\}$.
\end{enumerate}
Привести примеры, когда указанные включения являются строгими.
\end{task}

\begin{solution}[2]
По определению $\mathcal{P}(A) = \lbrace X : X \subseteq A \rbrace $, где X - подмножество A:
\begin{enumerate}
	\item[a)] $X \in \mathcal{P}(A \cap B) \Rightarrow X \in A \cap B \Rightarrow X \in A, X \in B \Rightarrow X \in \mathcal{P}(A) \cap \mathcal{P}(B)$;
	$X \in \mathcal{P}(A) \cap \mathcal{P}(B) \Rightarrow X \in \mathcal{P}(A), X \in \mathcal{P}(B) \Rightarrow X \in \mathcal{P}(A \cap B)$;
	\item[b)] $X \in \mathcal{P}(A) \cup \mathcal{P}(B) \Rightarrow X \in A$ or $X \in B \Rightarrow X \in \mathcal{P}(A \cup B)$;
	\item[v)] $X \in \mathcal{P}(A \setminus B) \Rightarrow X \in A $ and $ X \notin B \Rightarrow X \in (\mathcal{P}(A) \setminus \mathcal{P}(B)) \cup \lbrace \varnothing \rbrace$.
	\item[add)]
	\begin{enumerate}
	\item[b)]в случае если ни одно из множеств не пустое, то включение строгое
	\item[v)]eсли A - непустое множество, то выполняется строгость включения, а также когда А != Б.
	\end{enumerate}
\end{enumerate}
\end{solution}

% 3. Алфавиты и порядки
\begin{task}[3]
Пусть $A = \{a_1, \ldots, a_m\}$ — конечный алфавит, $A^n$ — множество слов длины $n$ в алфавите $A$.
\begin{enumerate}
    \item[(a)] На $A^n$ задано отношение $R_1$: для $v = a_{i_1}\ldots a_{i_n}$ и $w = a_{j_1}\ldots a_{j_n}$ положим $(v,w)\in R_1$ тогда и только тогда, когда $i_k\le j_k$ для всех $k=1,\dots,n$ и $i_k<j_k$ для некоторого $k$. Является ли $R_1$ отношением частичного (линейного) порядка?
    \item[(б)] На $A^*$ задано отношение $R_2$: для $v = a_{i_1}\ldots a_{i_n}$ и $w = a_{j_1}\ldots a_{j_r}$ положим $(v,w)\in R_2$ тогда и только тогда, когда существует $k$ от $1$ до $n$ с $i_\ell=j_\ell$ при $1\le \ell<k$ и $i_k<j_k$, причём первые $n$ символов $w$ совпадают со словом $v$. Является ли $R_2$ отношением частичного (линейного) порядка?
\end{enumerate}
\end{task}

\begin{solution}[3]
Чтобы отношение было частичным порядком в множестве необходимо выполнение 3 условий:
\begin{enumerate}
\item[1)] Рефлексивность: $\forall a \in A (a, a) \in R$
\item[2)] Транзитивность: $\forall a, b, c \in A: (a, b) \in R, (b, c) \in R \Rightarrow (a, c) \in R$
\item[3)] Антисимметричность: $\forall a, b \in A: (a, b) \in R, (b, a) \in R \Rightarrow a = b$
\end{enumerate}
Проверим отношения из задач на эти условий.
\begin{enumerate}
\item[а)] $(v, w) \in R \Leftrightarrow \forall k \in 1:n i_k \le j_k$ and $\exists k: i_k < j_k$
\begin{enumerate}
\item[1)] Сразу видно, что условие рефлексивности не выполняется, так как все буквы в словах одинаковые, то мы не найдём такой k, для которого $i_k$ и $j_k$ будут в отношении "меньше".
\item[2)] Выполняется
\item[3)] Не выполняется
\end{enumerate}
Мы поняли, что это строгий частичный порядок. Проверим, что отношение определено для всех двух элементво множества. Это отношение не определено, так как если $a_{i_1} > a_{j_1}$ и $a_{i_2} < a_{j_2}$, то они не будут в отношении находится.
\item[б)] $(v, w) \in R \Leftrightarrow \exists k \in 1:n i_l = j_l 1 \leqslant l < k, i_k < j_k $
\begin{enumerate}
\item[1)] рефлексивность не выполнена
\item[2)] транзитивность выполнена
\item[3)] антисимметричность не выполнена
\end{enumerate}
Следовательно, это частичный порядок. Проверим на линейный порядок. Это нелинейный порядок, так как не все слова являются частью другого.
\end{enumerate}
\end{solution}

% 4. Функции — область значений и свойства
\begin{task}[2]
Для каждой из функций найти область значений и указать, является ли функция инъективной, сюръективной, биекцией.
\begin{enumerate}
    \item[(а)] $f : \mathbb{R} \to \mathbb{R},\ f(x) = 3x + 1$;
    \item[(б)] $f : \mathbb{R} \to \mathbb{R},\ f(x) = x^2 + 1$;
    \item[(в)] $f : \mathbb{R} \to \mathbb{R},\ f(x) = x^3 - 1$;
    \item[(г)] $f : \mathbb{R} \to \mathbb{R},\ f(x) = e^x$;
    \item[(д)] $f : \mathbb{R} \to \mathbb{R},\ f(x) = \sqrt{3x^2 + 1}$;
    \item[(е)] $f : [-\pi/2, \pi/2] \to \mathbb{R},\ f(x) = \sin x$;
    \item[(ж)] $f : [0, \pi] \to \mathbb{R},\ f(x) = \sin x$;
    \item[(з)] $f : \mathbb{R} \to [-1, 1],\ f(x) = \sin x$;
    \item[(и)] $f : \mathbb{R} \to \mathbb{R},\ f(x) = x^2 \sin x$.
\end{enumerate}
\end{task}

\begin{solution}[4]
Что такое сюръекция, биекция и инъекция:
\begin{enumerate}
\item[1)] инъекция - отображение $x_1 = x_2 \Rightarrow f(x_1) = f(x_2)$
\item[2)] сюръекция - отображение $\forall y \exists x: f(x) = y$
\item[3)] биекция - это инъекция и сюръекция
\end{enumerate}
\begin{enumerate}
\item[1)] $E(f)=\mathbb{R}$, инъекция и сюръекция $\Rightarrow$ биекция
\item[2)] $E(f)=(1, +\infty)$, не инъекция, так как несколько $x$ дают 1 $y$, и не сюръекция, так как $y$ не поределен на отрицательных $\mathbb{R}$
\item[3)] $E(f)=\mathbb{R}$, инъекция и сюръекция $\Rightarrow$ биекция
\item[4)] $E(f)=(0, +\infty)$ инъекция и не сюръекция
\item[5)] $E(f)=[1, +\infty)$ не инъекция и не сюръекция
\item[6)] $E(f)=[-1, 1]$ инъекция, не сюръекция
\item[7)] $E(f)=[0, 1]$ не инъекция (в 0 и в $\pi$ значения равны 0), не сюръекция, так как $y$ находится только на отрезке от 0 до 1, а не во всей $\mathbb{R}$ 
\item[8)] $E(f)=[-1, 1]$, не инъекция, зато сюръекция
\item[9)] $E(f)=\mathbb{R}$ не инъекция, сюръекция
\end{enumerate}
\end{solution}

% 5. Композиция функций — логические следствия
\begin{task}[2]
Даны $g : A \to B$ и $f : B \to C$. Рассмотрим композицию $g\circ f : A \to C$, $(g\circ f)(x)=f(g(x))$. Определить, какие утверждения верны:
\begin{enumerate}
    \item[(а)] Если $g$ инъективна, то $g\circ f$ инъективна.
    \item[(б)] Если $f$ и $g$ сюръективны, то $g\circ f$ сюръективна.
    \item[(в)] Если $f$ и $g$ биекции, то $g\circ f$ биекция.
    \item[(г)] Если $g\circ f$ инъективна, то $f$ инъективна.
    \item[(д)] Если $g\circ f$ инъективна, то $g$ инъективна.
    \item[(е)] Если $g\circ f$ сюръективна, то $f$ сюръективна.
\end{enumerate}
\end{task}

\begin{solution}[5]
Здесь мы видим, что утверждения представляют из себя импликации. Следовательно, чтобы импликация давала ложь нужно истинность посылки и ложность заключения:
\begin{enumerate}
\item[а)] Неверно. Никто не наложил условий для отображения $f$, то есть в случае если $\exists b_1, b_2 \in B, g(a_1) = b_1, g(a_2) = b_2: b_1 \neq b_2$, то они могут совпасть во множестве $C: f(b_1) = f(b_2) \Rightarrow (g \circ f)(a_1) = (g \circ f)(a_2) \Rightarrow$ не выполняется инъективность, то есть композиция отображений не будет инъекцией.
\item[б)] Верно. $g$ - сюръекция $\Rightarrow \forall b \in B: \exists a \in A: g(a) = b$, $f$ - сюръекция $\Rightarrow \forall c \in C: \exists b \in B: f(b) = c \Rightarrow \forall c \in C: \exists a \in A: (g \circ f)(a) = c$. Значит, композиция сюръекций сюръективна.
\item[в)] Верно. Из всех $\forall a \in A$ мы переходим $\forall b \in B: g(a) = b$, дальше $\forall c \in C: f(b) = c$, то есть для $\forall a \in A: (g \circ f)(a) = c$. При этом выполняется условие инъекции.
\item[г)] Пусть $f$ - сюръекция. Это возможно, например, если $g$ инъекция, которая будет попадать только в те элементы из $B$, которые дают разные элементы $C$. Значит, это утверждение неверно.
\item[д)] Поступим также как и в предыдущем пункте: допустим, что $g$ - сюръекция (биекция нам не интересна, так как это инъекция). Если это так, тогда мы не сможем полностью утверждать посылку. $a_1, a_2 \in A, a_1 \neq a_2: g(a_1) = g(a_2) \Rightarrow f(g(a_1)) = f(g(a_2) \Rightarrow$ не выполняется инъекцтивность композиции. Следовательно, это утверждение верно
\item[е)] Пусть $f$ будет инъекцией. Значит, не вся область значений покрывается отображение $f \Rightarrow$ композиция не будет сюръекцией. Получаем верность утверждения
\end{enumerate}
\end{solution}
% 6. Кафе-мороженое (парадокс ссор)
\begin{task}[3]
Учащиеся одной школы часто собираются группами и ходят в кафе-мороженое. После такого посещения они ссорятся настолько, что никакие двое из них после этого вместе мороженое не едят. К концу года выяснилось, что в дальнейшем они могут ходить в кафе-мороженое только поодиночке. Докажите, что если число посещений было к этому времени больше 1, то оно не меньше числа учащихся в школе.
\end{task}
\begin{solution}[6]
Если рассматривать группу школьников, как пару школьников.
Для решения этой задачи хорошо подойдут графы. Пусть дети из школы - это вершины нашего графа, а походы в кофе-мороженое - это наши рёбра. Тогда нашу задачу можно переформулировать: если рёбер больше чем одно, то количество всех ребер больше или равно, чем количество всех школьников. А это очевидно. Рассмотрим простой пример. Пусть у нас $n$ школьников, то, для того чтобы связать одного школьника со всеми нужно $n - 1$ ребер и еще одно, так как мы рассматриваем граф из 3 вершин минимум (мы соединили 1 вершину 2 ребрами с двумя другими вершинами, а также соединили те две вершины между собой), нам необходимо как минимум $n$ походов в кафе.

\end{solution}

% 7. Вечерние визиты класса
\begin{task}[3]
30 учеников одного класса решили побывать друг у друга в гостях. Известно, что ученик за вечер может сделать несколько посещений, и что в тот вечер, когда к нему кто-нибудь должен прийти, он сам никуда не уходит. Покажите, что для того, чтобы все побывали в гостях у всех,
\begin{enumerate}
    \item[а)] четырёх вечеров недостаточно,
    \item[б)] пяти вечеров также недостаточно,
    \item[в)] а десяти вечеров достаточно,
    \item[г)] и даже семи вечеров тоже достаточно.
\end{enumerate}
\end{task}

% 8. Выборы мэра и знакомые
\begin{task}[3]
У каждого из жителей города $N$ число знакомых составляет не менее 30\% населения города. Житель идёт на выборы, если баллотируется хотя бы один из его знакомых. Докажите, что можно так провести выборы мэра города $N$ из двух кандидатов, что в них примет участие не менее половины жителей.
\end{task}

% 9. Комитеты в Думе
\begin{task}[3]
В Думе 1600 депутатов образовали 16000 комитетов по 80 человек в каждом. Докажите, что найдутся два комитета, имеющие не менее четырёх общих членов.
\end{task}

\vspace{1.5em}
\noindent\textbf{Примечание.}
\begin{quote}
Напоминание: задачи, имеющие сложность 1 должны уметь решать все. На решение этих задач даётся дедлайн – две недели (на первый раз 09.10.2025).
\end{quote}

\end{document}